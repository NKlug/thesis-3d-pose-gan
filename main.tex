% Preamble
\documentclass[11pt]{article}

% Packages
%\usepackage{a4wide}
\usepackage[a4paper, margin=2.22cm]{geometry}
\usepackage[english]{babel}
\usepackage[utf8]{inputenc}
\usepackage[hidelinks]{hyperref} % Makes citations clickable



\begin{document}
	\section{Protocols}
	\subsection{Protocol 1}
		\begin{itemize}
			\item Training: S1, S5, S6, S7, S8, S9 (commonly used, e.g. in \cite{chen17})
			\begin{itemize}
				\item \cite{drover18} do not use S9.
				\item The available code suggests that not every pose is used, but only such where at least one joint has moved by 40mm with respect to the former frame.
			\end{itemize} 			
			\item Evaluation: S11
			\begin{itemize}
				\item \cite{sun17} use every 64th frame (and also the code suggests this). If they only use the available code this would mean a frontal view.
				\item \cite{drover18} use ground truth 2d points for testing (might be the same as \cite{sun17})
				\item \cite{moreno-noguer16} use every 64th frame of the frontal view for testing
			\end{itemize}

			\item Error: \begin{itemize}
				\item \cite{drover18} use Mean per Joint Position Error (MPJPE) with scaling and rigid alignment to the ground truth skeleton (they don't mention which exactly)
				\item \cite{sun17} also use MPJPE after rigid alignment using Procrustes Analysis
				\item \cite{yasin16} use the MPJPE after alignment to the ground truth skeleton by a rigid transformation (they don't mention which exactly)
				\item \cite{kostrikov14} align the poses by a rigid transformation using \emph{least squares} before computing the MPJPE error
				\item \cite{tome17} perform a similarity transformation using a Procrustes Analysis
			\end{itemize}
			
		\end{itemize}
	\subsection{Protocol 2}
		\begin{itemize}
			\item Training: S1, S5, S6, S7, S8, S9
			\begin{itemize}
				\item Similar to protocol 1 the code suggests that not every pose is used, but only such where at least one joint has moved by 40mm with respect to the former frame.
				\item \cite{bogo16} evaluate on five different action sequences captured from the frontal camera
				\item \cite{tekin16, tekin17} use all camera views for training
			\end{itemize}
			\item Evaluation: S9, S11 
			\begin{itemize}
				\item \cite{sun17} use every 64th frame
				\item \cite{tekin16, tekin17} use all camera views for testing
				\item \cite{moreno-noguer16} say they use all images for testing (whatever this means) 
			\end{itemize}
			\item Error: \begin{itemize}
				\item \cite{sun17} use MPJPE apparently without any alignment and scaling
				\item \cite{martinez17} align the root (central hip), they do not mention any scaling
				\item \cite{tome17} do not mention any alignment
				\item \cite{zhou18} scale the output such that the mean limb length is identical to the average value of all training subjects and align the root locations. Procrustes Analysis is not allowed.
				\item \cite{bogo16} apply a similarity transformation to align the reconstructed 3D joints via the Procrustes analysis on every frame
				\item \cite{zhou16} align the root locations and scale the output such that the mean limb length is identical to the average value of all training subjects. Procrustes alignment is not allowed.
				\item it seems that \cite{tekin16} also align the root nodes
				\item \cite{tekin17} use a Procrustes transformation before measuring the MPJPE
				\item \cite{pavlakos17} align the root joints
			\end{itemize}
		\end{itemize}
	
	\subsection{Miscellaneous}
	\begin{itemize}
		\item \cite{jahangiri17} evaluate using all subjects after a similarity transformation obtained by Procrustes alignment
	\end{itemize}
	
	\section{Questions and Problems}
	\begin{itemize}
		\item It is sometimes not clear which 2d poses are used for testing.
		\item The transformations applied before calculating the error are not consistent within the protocols.
	\end{itemize}
	
	
	\section{Our approach}
		\subsection{Protocol 1}
			\begin{itemize}
				\item Training subjects: S1, S5, S6, S7, S8, S9
				\item Testing/Evaluation subjects: S11
				\item For now we train on all frames available, \emph{not} such where at least one joint has moved by 40mm
				\item For evaluation every 64th frame available for the according subjects is used
				\item Error metric: MPJPE
				\item Before calculating the error, the poses are transformed using a Procrustes Transformation and the root nodes are aligned
			\end{itemize}
		\subsection{Protocol 2}
			\begin{itemize}
				\item Training subjects: S1, S5, S6, S7, S8
				\item Testing/Evaluation subjects: S9, S11
				\item For now we train on all frames available, \emph{not} such where at least one joint has moved by 40mm
				\item For evaluation every 64th frame available for the according subjects is used
				\item Error metric: MPJPE
				\item Before calculating the error, the root nodes are aligned
			\end{itemize}
	
  \bibliography{bibliography}
  \bibliographystyle{apalike}
  
\end{document}