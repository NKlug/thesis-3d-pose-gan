\begin{figure}
	\centering
	\begin{tikzpicture}
	[
	 block/.style ={rectangle, draw=black, thick, text width=20em, align=center, minimum height=2em}
	]
	\node[] (a) [block] {Fully Connected Layer (1024)};
	\node[below= -1.5\pgflinewidth of a] (b) [block] {ReLU};
	\node[below= -1.5\pgflinewidth of b] (c) [block] {Fully Connected Layer (1024)};
	\node[below= -1.5\pgflinewidth of c] (d) [block] {ReLU};
	\node[below= -1.5\pgflinewidth of d] (e) [block] {$\bigoplus$};
	\node[above=of a] (x) [] {};
	\draw[->, line width=1pt] (x) -- (a);
	\node[below=of e] (y) [] {};
	\draw[->, line width=1pt] (e) -- (y);
	\node[above=.4 of a] (z) [] {};
	\node[right=12em of z] (h) [] {};
	\draw[line width=1pt] (z.center) -- (h.center);
	\draw[->, line width=1pt] (h.center) |- (e.east);
	\end{tikzpicture}
	\caption{Architecture of the Residual Blocks in generator and discriminator.
	Two fully connected layers of size 1024 are each followed by a Rectified Linear Unit (ReLU). After the last ReLU the input is added to the output.
	\citet{drover18} suggest a Batch Normalization immediately after each Fully Connected Layer, which has not been found to work.}
	\label{fig:residual-block}
\end{figure}