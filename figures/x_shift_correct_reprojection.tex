\begin{figure}
		\centering
		\begin{tikzpicture}[trim axis left]
		\begin{axis}[
		tiny,
		width=\textwidth,
		height=\axisdefaultheight,
		xlabel={$dx$ [m]},
		%			axis lines = middle,
		y label style={at={(axis description cs:0.13,1.1)},rotate=-90,anchor=north},
		ylabel={MPJPE [mm]},
		grid=both,
		enlarge x limits = false,
		grid style={draw=gray!50},
		minor tick num = 1,
		tick style={draw=gray!50},
		xtick={-10,-9,...,10},
		extra x ticks={0},
		scaled x ticks = false,
		no markers,
		every axis plot/.append style={}
		]
		\addplot +[restrict expr to domain={\coordindex}{90:491}] table [x=a, y=b, col sep=comma] {figures/plot_e_03_07_original_x_shift.csv};
		\addplot[color=black!60!green] table [x=a, y=b, col sep=comma] {figures/plot_e_01_06_shifted_generator_x_shift.csv};
		\addplot[color=red] table [x=a, y=b, col sep=comma] {figures/plot_e_11_08_default_generator_x_shift.csv};
		\addplot[color=orange] table [x=a, y=b, col sep=comma] {figures/plot_e_03_07_original_correct_reprojection_x_shift.csv};
		\end{axis}
		\end{tikzpicture}
	\caption{
		Comparison of MPJPEs for different networks. A pre-trained version of the original system with correct re-projection of the 2D poses (orange) outperforms the original system (blue).
		Training the former from ground up (red) reduces the MPJPE significantly, especially for larger offsets $dx$.
		If the generator with correct re-projection is trained with poses only normalized in scale, but not in location, the MPJPE decreases even more (green).
		The MPJPEs are calculated using Protocol 2.
		The distance between the camera and the poses is approximately 5m.}
	\label{fig:x-shift-error-correct-reproj}
	\Todo[inline]{Use correct data for shifted generator (data with cross entropy loss)}
\end{figure}