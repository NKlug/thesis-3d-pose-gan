\begin{figure}
	\begin{subfigure}[]{.3\textwidth}
		\centering
		\begin{tikzpicture}[trim axis left]
		\begin{axis}[
			tiny,
			width=1.25\textwidth,
			xlabel={$dx$ [m]},
%			axis lines = middle,
    		y label style={at={(axis description cs:0.35,1.15)},rotate=-90,anchor=north},
			ylabel={MPJPE [mm]},
			grid=both,
			enlarge x limits = false,
			grid style={draw=gray!50},
			minor tick num = 1,
			tick style={draw=gray!50},
			xtick={-100,-50,0,50,100},
			extra x ticks={0},
			scaled x ticks = false,
			no markers,
			every axis plot/.append style={}
		]
		\addplot table [x=a, y=b, col sep=comma] {figures/plot_e_03_07_original_x_shift.csv};
		\addplot table [x=a, y=b, col sep=comma] {figures/plot_theoretical_x_shift.csv};
		\end{axis}
		\end{tikzpicture}
		\subcaption{unscaled}
	\end{subfigure}\hfill
	\begin{subfigure}[]{.3\textwidth}
		\centering
		\begin{tikzpicture}[trim axis left]
		\begin{axis}[
						tiny,
			width=1.25\textwidth,
			xlabel={$dx$ [m]},
			%			axis lines = middle,
			y label style={at={(axis description cs:0.35,1.15)},rotate=-90,anchor=north},
			ylabel={MPJPE [mm]},
			grid=both,
			enlarge x limits = false,
			grid style={draw=gray!50},
			minor tick num = 1,
			tick style={draw=gray!50},
			xtick={-100,-50,0,50,100},
			extra x ticks={0},
			scaled x ticks = false,
			no markers,
			every axis plot/.append style={}
		]
		\addplot table [x=a, y=b, col sep=comma] {figures/plot_e_03_07_original_x_shift.csv};
		\addplot table [x=a, y=b, col sep=comma] {figures/plot_theoretical_x_shift_scaled.csv};
		\end{axis}
		\end{tikzpicture}
		\subcaption{scaled}
	\end{subfigure}\hfill
	\begin{subfigure}[]{.3\textwidth}
		\centering
		\begin{tikzpicture}[trim axis left]
			\begin{axis}[
				tiny,
				width=1.25\textwidth,
				xlabel={$dx$ [m]},
				%			axis lines = middle,
				y label style={at={(axis description cs:0.35,1.15)},rotate=-90,anchor=north},
				ylabel={MPJPE [mm]},
				grid=both,
				enlarge x limits = false,
				grid style={draw=gray!50},
				minor tick num = 1,
				tick style={draw=gray!50},
				xtick={-3,-2,-1,0,1,2,3},
				extra x ticks={0},
				scaled x ticks = false,
				no markers,
				every axis plot/.append style={}
				]
				\addplot +[restrict expr to domain={\coordindex}{180:401}] table [x=a, y=b, col sep=comma] {figures/plot_e_03_07_original_x_shift.csv};
				\addplot +[restrict expr to domain={\coordindex}{180:401}] table [x=a, y=b, col sep=comma] {figures/plot_theoretical_x_shift_scaled.csv};
			\end{axis}
		\end{tikzpicture}
		\subcaption{scaled}
	\end{subfigure}
			

	
	\caption{Theoretical (red) and experimental (blue) MPJPEs for different values of $dx$ for \textbf{Protocol 2}.
		The unscaled theoretical minimal MPJPE does not represent a lower bound (a), meanwhile the one with scaled joint coordinates does (b and c). 
		The theoretical errors are obtained by calculating $\Delta d$ for every pose in the Protocol 2 test data.
		The distance between the camera and the poses is ten times the norm limb length.
		This results in an average camera distance of approximately $5$m.}
	\label{fig:x-shift-error}
\end{figure}