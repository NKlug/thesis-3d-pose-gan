\section{Evaluation on Different Datasets}
\label{sec:evaluation}

In this section the model's performance will be analyzed for different configurations of training and test data.
For training the Adam Optimizer \cite{kingma17} with an initial learning rate of $0.0002$ and $\beta_1 = 0.5$ is used for both generator and discriminator.
This particular choice of $\beta_1$ has shown to result in faster convergence.
As in the work by \citet{drover18}, the data is split up into batches of 32768 poses for training.
Generator and discriminator are trained alternately with the same batch of poses.
Before adjusting the networks' parameters, adaptive clipping is applied to the gradients \cite[Section~3.2.1]{chorowski14}.


\subsection{Results for Synthetic Data}

Later in this work, the effects of pose normalization will be analyzed.
For a meaningful comparison of normalized and non-normalized poses, in a first step the results for poses which are naturally normalized are obtained.
The normalization applied to the 2D poses consists of two parts: Centering the pose node and scaling.
As will further be elaborated in Section \ref{sec:effects-of-normalization}, the need for scaling can hardly be avoided.
Centering however can be easily avoided, when the camera photographing the 3D poses already looks at the center of the pose (the root joint).

For a fine control of the 2D poses' degree of normalization the 2D poses included in the dataset are not viable.
Hence, 2D poses are synthetically generated by projecting the monocular 3D poses available in the Human3.6M dataset.
These synthetic poses are created by "photographing" them with virtual cameras that provide the desired 2D poses.
For training and testing, cameras with integer azimuth and elevation angles randomly sampled from [0, 359] degrees and [0, 20] degrees are used.
This choice of elevation angles is based on the heuristic that in reality people are rather photographed from above than from below.
The 3D poses are first normalized such that the norm limb has length 1 and then photographed with a camera-root-joint distance of 10 units.
Thus and from the specific choice of elevation angles, in most cases the projected root limb has a length of almost 0.1 and only slight scaling is necessary.
For training, \citet{drover18} follow a similar procedure and augment synthetic 2D poses, although they use 8 fixed cameras instead of randomly sampling camera angles for each pose.

\begin{table}[]	
	\centering
	\begin{tabularx}{\textwidth}{l *{8}{Y}}
		\toprule
		Method & Direct. & Discuss & Eat & Greet & Phone & Pose & Purchase & Sit \\
		\midrule
		\citet{drover18} & 34.3 & 36.4 & 28.4 & 33.7 & 30.0 & 43.8 & 31.7 & 32.5\\
		Synthetic & 36.3 & 35.5 & 35.6 & 42.6 & 34.8 & 44.1 & 45.2 & 36.3 \\
		Human3.6M & 35.7 & 34.2 & 37.9 & 40.3 & 37.3 & 38.6 & 43.5 & 40.4 \\
		\bottomrule
		\toprule
		Method & SitDown & Smoke & TPhoto & Wait & Walk & WDog & WTog. & \textbf{Avg.}\\
		\midrule
		\citet{drover18} & 48.9 & 32.1 & 43.8 & 36.0 & 25.1 & 34.1 & 30.3 & \textbf{34.2}\\
		Synthetic & 51.9 & 41.9 & 50.8 & 43.0 & 38.5 & 49.6 & 40.8 & \textbf{41.0} \\
		Human3.6M & 59.7 & 45.0 & 52.6 & 42.2 & 32.5 & 45.8 & 36.0 & \textbf{41.2} \\
		\bottomrule
	\end{tabularx}
	\caption{
		Comparison of the MPJPEs reported by \citet{drover18} and for a rebuilt system trained with synthetic data. 
		The rebuilt system is tested with synthetic data ("Synthetic") and 2D poses from Human3.6M  \cite{ionescu14} ("Human3.6M").
		The results were obtained using \textbf{Protocol 1}. The MPJPEs are given in millimeters.
	 }
	\label{tbl:results-original-protocol1}
	\info[inline]{Augmented obtained with best.ckpt-16161 from e\_03\_07\_original.json;
		Human3.6M obtained with real 2D poses with best.ckpt-16161 from e\_03\_07\_original.json}
\end{table}
\begin{table}[]	
	\centering
	\begin{tabularx}{\textwidth}{l *{8}{Y}}
		\toprule
		Method & Direct. & Discuss & Eat & Greet & Phone & Pose & Purchase & Sit \\
		\midrule
		Synthetic & 48.8 & 51.5 & 41.5 & 57.7 & 49.4 & 54.3 & 48.8 & 50.0 \\
		Human3.6M & 86.5 & 81.7 & 65.2 & 86.6 & 82.4 & 85.2 & 93.6 & 61.3 \\
		\bottomrule
		\toprule
		Method & SitDown & Smoke & TPhoto & Wait & Walk & WDog & WTog. & \textbf{Avg.}\\
		\midrule
		Synthetic & 64.3 & 51.9 & 64.6 & 57.1 & 54.3 & 57.9 & 53.2 & \textbf{54.8} \\
		Human3.6M & 82.3 & 73.8 & 93.2 & 85.6 & 78.3 & 80.9 & 82.1 & \textbf{80.4} \\
		\bottomrule
	\end{tabularx}
	\caption{
		Comparison of the MPJPEs of the replicated system trained and tested with synthetic data and 2D poses from the Human3.6M dataset \cite{ionescu14}. 
		For \cite{drover18} only results with rigid alignment are available, which allows no fair comparison.
		The results were obtained using \textbf{Protocol 2}. The MPJPEs are given in millimeters.
	 }
	\label{tbl:results-original-protocol2}
	\info[inline]{Synthetic obtained with best.ckpt-16161 from e\_03\_07\_original.json;
	Human3.6M obtained with real 2D poses with best.ckpt-16161 from e\_03\_07\_original}
\end{table}
\begin{table}	
	\centering
	\begin{tabularx}{\textwidth}{l *{4}{Y}}
		\toprule
		Method & Acting3 & Freestyle3 & Walking2 & \textbf{Average}\\
		\midrule
		Protocol 1 & 65.5 & 74.7 & 65.7 & \textbf{68.3} \\
		Protocol 2 & 98.8 & 105.8 &  96.8 & \textbf{100.2} \\
		\bottomrule
	\end{tabularx}
	\caption{
		MPJPEs for the test data of the \textbf{TotalCapture} dataset \cite{trumble17}. The system was trained with synthetic data from Human3.6M.
		The results were obtained using the transformations of Protocol 1 and 2 as indicated. The MPJPEs are given in millimeters.
	 }
	\label{tbl:results-original-totalcapture}
	\info[inline]{Protocol 1 obtained with best.ckpt-16161 from e\_03\_07\_original.json;
	Protocol 2 obtained with best.ckpt-16161 from e\_03\_07\_original.json}
\end{table}

Table \ref{tbl:results-original-protocol1} shows the results for \textbf{Protocol 1} reported in \cite{drover18} and for the replicated system trained with synthetic data.
For the evaluation of the latter, one synthetic 2D poses is deterministically sampled from each monocular 3D pose available for subject 11.
The results for the synthetic training data can not compete with the results presented by \citet{drover18}.
The exact source of the 6.8mm average discrepancy is not clear; the reasons can only be speculated about.
\citet{drover18} don't mention whether they also use synthetic poses for testing or evaluate only on 2D poses provided in Human3.6M.
Before calculating the MPJPE, they apply a similarity transformation, but don't explicitly mention the Procrustes Analysis used in this work.
Further discrepancy can also arise from different internal configurations, like the exact training procedure or the applied gradient clipping.
Finally, extensive mining for the best performing model might also be a reason for the lower MPJPE.

Evaluation results for the monocular 2D poses in Human3.6M are also displayed in Table \ref{tbl:results-original-protocol1}.
The MPJPEs for the different categories are slightly worse than the ones for the synthetic data.
This comes as no surprise, as most machine learning systems will perform better on test data which is more similar to the training data.
For the synthetic testing data, this is certainly the case.

\Todo{Add correct difference}
The results for \textbf{Protocol 2} are given in Table \ref{tbl:results-original-protocol2}.
As rotation is not allowed for aligning the poses, the MPJPEs are approximately 88.8mm worse than those for Protocol 1.
Overall, the same relations between the synthetic poses and the poses from Human3.6M can be seen.
As \cite{drover18} also allow rigid alignment for Protocol 2, their results are not directly comparable.

On the test data of the TotalCapture dataset, the average MPJPE for Protocol 1 is 68.3mm and for Protocol 2 100.2mm respectively (Table \ref{tbl:results-original-totalcapture}).
Apart from the aforementioned phenomenon, the key points of the joints are slightly different from the ones in Human3.6M.
Especially the angle between the "hip bones" is noticeable.
In Human3.6M, this angle is 180 degrees and the spine is perpendicular to the hips, whereas in TotalCapture, the angle is substantially smaller than 180 degrees (see Figure \ref{fig:human-totalcapture}).
Although the results are slightly worse for TotalCapture, they still show that the system generalizes to unseen data in a reasonable way.

\begin{figure}
	\Todo[inline]{Add figure!}
	\caption{Comparison of Human3.6M and TotalCapture poses}
	\label{fig:human-totalcapture}
\end{figure}
\Todo{Add a few qualitative results.}

Protocol 2: \citet{drover18} do evaluate for Protocol 2 but use rigid alignment which does not allow fair comparison with the actual Protocol 2.


\Todo[inline]{Compare sampled data error and error on real 2D poses from the dataset. TotalCapture: Angle between hip bones different} 
\subsection{Different errors for different sets of joints}
Using only 15 joints yields a much lower MPJPE than using 32 joints. Reasons: ...	
From now on all numbers are calculated for 15 joint poses.

\subsection{Training with 1:1 mix of augmented and real data}
Important: Generator should output the same "kind" of data as the "real" data the discriminator receives.
Improvements of evaluation on real 2D data. 
This shows that the system is expected to perform well in scenarios where it is trained with data it should later infer.
Results for original data get better

\Todo{"The reasons for the additional error of the real 2D data (shifting) will be extensively discussed in Section \ref{sec:effects-of-normalization}}

\subsection{Training with augmented cameras similar to the real cameras}
Performance of system with cameras similar to the real cameras.
Results for real 2D data.