\section{Improving the MPJPE using limb lengths}

In a standard fully-supervised network, training is heavily dependent on ground truth data.
As the key feature of the system described in this thesis is that it is only trained with 2D poses, without 3D ground truth poses, there is not much additional information that can be given to the system for improved results.
For training, the generator only receives information about the distribution to be learned through the error from the discriminator, which on his part only receives elements from the real data distribution.
Without 3D ground truth poses, the only additional information that can be given to the system originates from the kinematics and the structure of human 3D poses.
The angles between certain limbs can only have certain values, e.g. an arm can only be folded to some extent behind one's back. 
Even without any information about the kinematics, the generator already produces reasonably looking 3D poses that seem to have sensible limb angles, so simple restrictions for those are not expected to bring any advantage.
Describing the exact kinematics of human 3D poses would become quite complex very quickly is out of the scope of this work.
A fact that can be made use of much more easily comes from the basic structure of the human skeleton, more precisely its symmetry. 
There are certain limbs the human body which can be safely assumed to have the equal lengths for all humans, such as the upper and lower arms and legs.


Modification of GAN loss function for generator:
The discriminator loss is the same as in equation \ref{eq:discriminator-loss}. 
An additional term is added to the generator loss. Let a \textit{limb} $l = (u, v)$ be the connection between joints $u$ and $v$ and $S \coloneqq \{(l_1, l_2)~|~ l_1, l_2~\text{are limbs in a skeleton and have equal lengths}\}$ the set of all symmetric limbs. 
The generator loss with symmetric limb error basically takes into account the difference in length of two symmetric limbs.
Formally, it is defined as \Todo{Find proper formulation of limbs: symmetric limbs are defined for the skeleton, but the error is calculated for the predicted points}
\begin{equation}
	loss_G = -\mathbb{E}_{p\sim P_{fake}}(\log(D(p))) 
	+ \frac{1}{|S|}\sum_{((u_1, v_1), (u_2, v_2)) \in S} \bigl\lvert \norm{u_1 - v_1}_2 - \norm{u_2 - v_2}_2 \bigr\rvert \ .
\end{equation}
In practice, the symmetric limbs for the 15 joint skeleton are the upper arms, the lower arms, the upper legs, the lower legs, the hips (distance from left and right hip to the root joint) and the shoulders (distance from shoulders to neck).