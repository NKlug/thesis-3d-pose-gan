\section{Introduction}

Human pose estimation has gained a lot of attention in the last years.
With the emergence of models that estimate poses from RGB images only, poses estimation has become more accessible than ever.


Why human pose estimation?
Human Pose Estimation has various applications.



\Todo[inline]{2D/3D = two/three dimensional}

\begin{itemize}
	\item applications: surveillance in work field, animationen and visualizations, sports, medicine -> tracking of progress
	\item History: 2D pose estimation systems have made big progress -> Motivation of 2D to 3D lifting
	\item definition pose
	\item definition of the task
	\item Comparison to \citet{wandt19}
	\item Medical applications \cite{aroeira16, khan18}
\end{itemize}

\begin{itemize}
	\item Common 3D datasets with 3D annotation. Very few annotated. Human3.6M, HumanEvaI/II, TotalCapture, SURREAL (synthetic), KTH Multiview Football II
	\item Motivation for training with 2D data only
\end{itemize}

notoriously
Whereas there are a lot of annotated human 2D pose estimation datasets (MPII, COCO Keypoints, PoseTrack, DensePose), there are only very few annotated 3D pose datasets. 
2D poses can be easily captured and annotated.
Only images are necessary for this.
As opposed to this, capturing ground truth data for 3D poses is notoriously harder.
For precise ground truth poses, a motion capture system is required which is not broadly available due to the high cost.
Though, this saves the effort of manual labeling.


\emph{A (three dimensional) \textit{pose} is a collection of $n$ points in $\mathbb{R}^3$.}
Throughout this work coordinates of 3D points will be denoted by capital letters and 2D points by lower case letters.

In this work the GAN based 3D human pose estimation system by \citet{drover18} will be analyzed.

\subsection{2D and 3D pose estimation}
Why 2D to 3D?


\subsection{Outline}

\autoref{sec:network} begins with a short introduction into Generative Adversarial Networks and perspective projection.
Those topics are the core of the 3D human pose estimation system proposed by \citet{drover18} that are explained subsequently.

In \autoref{sec:data} the datasets used for evaluation -- which is mainly the Human3.6M dataset \cite{ionescu14} -- are presented.
As most pose estimation systems do not estimate absolute 3D poses, the estimated poses have to be transformed to be comparable to the ground truth poses.
The two main protocols that evolved in literature allow the application of different types of transformations to the estimated poses.
Those protocols are analyzed and discussed.

Afterwards, in \autoref{sec:evaluation} baseline errors for a rebuilt version of the system is established.
Those are needed for the analysis of the effects of pose normalization later.
The system is evaluated for synthetically generated 2D poses and monocular 2D poses included in the Human3.6M dataset.
In order to get insights about the systems ability to generalize to unknown data, the system is also evaluated with human poses from the TotalCapture dataset \cite{trumble17}.

In \autoref{sec:loss-function-modification} a modified loss function for the generator is presented.
The new loss passes high-level knowledge about human poses to the generator. who can use this information to refine the estimated 3D poses.

\autoref{sec:effects-of-normalization} is the main part of this thesis.
Here, the two normalization constraints the 2D input poses have to fulfill in the 3D pose estimation system discussed in \autoref{sec:network} will be analyzed.
The 2D input poses are expected to be normalized in location and scale.
By re-constructing the projections and re-projections of the poses during the estimation process, a theoretical lower bound for the minimal error is derived for both types of normalization.
Experiments are conducted in order to confirm the so-found lower bounds.

The results of the previous section show that especially for the normalization in location introduces a significant additional error.
In \autoref{sec:network-adjusting} the system by \citet{drover18} is modified and trained such that the effects of normalization in location are diminished.

\autoref{sec:conclusion} concludes the thesis. It provides a summary of the results.
