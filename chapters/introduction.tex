\section{Introduction}

\Todo[inline]{2D/3D = two/three dimensional}

\begin{itemize}
	\item applications: surveillance in work field, animationen and visualizations, sports, medicine -> tracking of progress
	\item History: 2D pose estimation systems have made big progress -> Motivation of 2D to 3D lifting
	\item definition pose
	\item definition of the task
	\item Comparison to \citet{wandt19}
\end{itemize}

In this work the GAN based 3D human pose estimation system by \citet{drover18} will be analyzed.

\subsection{2D and 3D pose estimation}

\subsection{Outline}
This thesis has the following chapters \dots + short description!


\begin{itemize}
	\item Common 3D datasets with 3D annotation. Very few annotated. Human3.6M, HumanEvaI/II, TotalCapture, SURREAL (synthetic), KTH Multiview Football II
	\item Motivation for training with 2D data only
\end{itemize}

notoriously
Whereas there are a lot of annotated human 2D pose estimation datasets (MPII, COCO Keypoints, PoseTrack, DensePose), there are only very few annotated 3D pose datasets. 
2D poses can be easily captured and annotated.
Only images are necessary for this.
As opposed to this, capturing ground truth data for 3D poses is notoriously harder.
For precise ground truth poses, a motion capture system is required which is not broadly available due to the high cost.
Though, this saves the effort of manual labeling.


A (three dimensional) \textit{pose} is a collection of $n$ points in $\mathbb{R}^3$.
Throughout this work coordinates of 3D points will be denoted by capital letters and 2D points by lower case letters.
