\section{Prevention of the Effects of Pose Normalization}
\label{sec:network-adjusting}

The results of the previous section show that the network proposed by \citet{drover18} is not viable for practical use as is.
Especially if the camera photographing the poses is not centered at the root joint and the root joint thus is not naturally aligned with the origin of the image plane, a significant increase in MPJPE is measurable.
This motivates further looking into how this increased error can be circumvented and potentially modifying the network.
This section will focus on the error introduced by shifting along the image plane as it shows a far more significant increase in MPJPE than shifting along the z axis.
Again, for simplicity, only shifting along the x axis will be considered.

In \autoref{eq:re-projected-X}, if the 2D points would be re-projected correctly (i.e. shifted back before re-projection), in theory the lower bound in \autoref{eq:delta-d} would no longer apply.
In that case, if the generator estimates all depths $\widetilde{Z}_i$ perfectly (that is $\widetilde{Z}_i = Z_i$) the re-projected pose equals the ground truth pose up to a shift.
Thus, in a first attempt to efface the increased error, the re-projection of the poses into three dimensional space is adapted.

\subsection{Generator with Correct Re-Projection}

For a correct re-projection of the 2D poses into three dimensions, the network presented earlier is slightly modified.
The network now receives poses that are only normalized in scale, but not in location.
For an input pose $p$, in a first step, the vector $(dx, dy)$ given by the root joint's coordinates is subtracted from all 2D points, such that the 2D pose is now normalized both in scale and location.
Afterwards, the depth offsets $o_i$ are estimated for each point in the same way as in the original network.
Before the pose is then re-projected into three dimensions, the 2D points are shifted back by adding $(dx, dy)$ to each point again.
As explained above, this procedure theoretically circumvents the lower bound found in the previous section.
For the evaluation of this modified network, a pre-trained version of the original network in combination with the correct re-projection is used.
The Protocol 2 MPJPEs are again calculated for different offsets $dx$, while $dy = 0$.
The results for this procedure are shown in \autoref{fig:x-shift-error-correct-reproj} (orange graph).
The MPJPE decreased notably compared to the original network (blue graph), but still increases significantly for increasing offsets $dx$.

In another attempt to reduce the MPJPE, the modified network presented above is trained from ground up.
Similar to \autoref{sec:results-augmented}, the discriminator still receives poses which are already naturally normalized in location.
Since the generator now produces 3D poses which are not necessarily aligned with the origin of the coordinate system, but may have offsets in x or y dimension, the poses have to be normalized.
For this, the 3D poses' root joints are aligned with the origin, and then fed to the Random Projection Layer as before.
The system is trained with poses having varying offsets uniformly drawn from $[-10\text{m}, 10\text{m}]$ and the same hyperparameters as in \autoref{sec:evaluation}.

For an offset of $dx = 0$ the MPJPE is slightly higher than for the original network ($59.1$mm vs. $54.8$mm).
Despite that, for larger offsets along the x axis, the MPJPE is now reduced significantly.
A plot of the resulting curve is depicted in \autoref{fig:x-shift-error-correct-reproj} (red graph).
For a camera angle of approximately $63^{\circ}$ ($10$m offset at a camera distance of $5$m) the MPJPE now is only at $82$mm, compared to the previous $350$mm.

\begin{figure}
		\centering
		\begin{tikzpicture}[trim axis left]
		\begin{axis}[
		tiny,
		width=\textwidth,
		height=\axisdefaultheight,
		xlabel={$dx$ [m]},
		%			axis lines = middle,
		y label style={at={(axis description cs:0.13,1.1)},rotate=-90,anchor=north},
		ylabel={MPJPE [mm]},
		grid=both,
		enlarge x limits = false,
		grid style={draw=gray!50},
		minor tick num = 1,
		tick style={draw=gray!50},
		xtick={-10,-9,...,10},
		extra x ticks={0},
		scaled x ticks = false,
		no markers,
		every axis plot/.append style={}
		]
		\addplot +[restrict expr to domain={\coordindex}{90:491}] table [x=a, y=b, col sep=comma] {figures/plot_e_03_07_original_x_shift.csv};
		\addplot[color=black!60!green] table [x=a, y=b, col sep=comma] {figures/plot_e_01_06_shifted_generator_x_shift.csv};
		\addplot[color=red] table [x=a, y=b, col sep=comma] {figures/plot_e_11_08_default_generator_x_shift.csv};
		\addplot[color=orange] table [x=a, y=b, col sep=comma] {figures/plot_e_03_07_original_correct_reprojection_x_shift.csv};
		\end{axis}
		\end{tikzpicture}
	\caption{
		Comparison of MPJPEs for different networks. A pre-trained version of the original system with correct re-projection of the 2D poses (orange) outperforms the original system (blue).
		Training the former from ground up (red) reduces the MPJPE significantly, especially for larger offsets $dx$.
		If the generator with correct re-projection is trained with poses only normalized in scale, but not in location, the MPJPE decreases even more (green).
		The MPJPEs are calculated using Protocol 2.
		The distance between the camera and the poses is approximately 5m.}
	\label{fig:x-shift-error-correct-reproj}
	\Todo[inline]{Use correct data for shifted generator (data with cross entropy loss)}
\end{figure}

\subsection{Generator without Root Joint Alignment}

With only minor modifications to the system, the results already improved remarkably.
The correct re-projection of the poses into three dimensions is able to invalidate the lower bound found in \autoref{sec:x-shift-error}.
However, ambiguities between poses shifted by different offsets still exist, as the generator still only receives normalized poses.
In order to eliminate those ambiguities, an additional change has to be made to the generator.

Instead of poses normalized in scale and location, the generator now only receives poses that are normalized in scale.
That way, it can differentiate between poses photographed with different offsets from the center.
Again, the system is trained with poses with offsets $dx$ in $[-10\text{m}, 10\text{m}]$ while for $dy$ is still $0$.
As before, the 2D poses are re-projected correctly into three dimensions.
The hyperparameters are the same as in \autoref{sec:evaluation}.
Note that while prior to now, the root joint could have been left out during training (a constant input of $0$ has no effect on the neural network), it is now important to include it.

\Todo{Validate those results.}
Experimental results are displayed in \autoref{fig:x-shift-error-correct-reproj} (green graph).
The MPJPE could again be reduced and is now at $70$mm for a $10$m x offset at a camera distance of $5$m.
Compared to the results of the previous section, that is an improvement of approximately $10$mm on average.
At $dx = 0$, the results now equal those for the original system (up to $1$mm).

%Recalling \autoref{sec:results-augmented}, the results for Protocol 2 for the monocular 2D poses included in the Human3.6M dataset could not compete with those for synthetic data.
%The lowest achievable average MPJPE was $72.5$mm.
%One possible explanation for that was that Human3.6M's pose were not naturally aligned with the origin of the image plane.
%With the modified network, this problem should now be solved.
%In fact, the average Protocol 2 MPJPE for those 2D poses could now be reduced to \dots.

These results conclude the section, and the thesis.
The next and last section provides a summary of the results from each chapter.
