\section{Analysis of the Effects of Pose Normalization}
\label{sec:effects-of-normalization}

In \autoref{sec:evaluation} the MPJPE for the monocular 2D poses included in the Human3.6M dataset was noticeably higher than the MPJPE for the synthetically generated poses.
One reason for this can be the transformations that have to be applied to Human3.6M's 2D poses in order to satisfy the constraints for the input poses described in \autoref{sec:network}.
Those are:
\begin{enumerate}[label=(\Alph*)]
	\item The 2D input pose's root joint is centered at the origin.
	\item A designated norm limb has length 0.1.
\end{enumerate}

In reality, both constraints are rarely satisfied from the start.
Thus, the input 2D poses are normalized by shifting and scaling before being fed to the generator.
In this chapter it will be shown both theoretically and experimentally that this approach inevitably leads to additional errors being introduced into the system.

In the following, the camera used for projection is located at the origin of the coordinate system and looks into positive z direction (see \autoref{fig:camera-projection-setup}).
The generator $G$ will be regarded as a black box that takes a normalized 2D pose as input and outputs the depths (the z coordinates) of the input points which then can be re-projected into three dimensional space following \autoref{eq:perspective-re-projection}.

\subsection{Shifting along one Image Plane Axis}
\label{sec:x-shift-error}
First, constraint (A) will be analyzed.
The basic approach here is the following:
Let $P$ be a 3D pose which is centered at the origin of the x-y-plane at depth $Z$.
$P$ is shifted by the vector $(dx, dy, 0)$ along the x-y-plane and projected into two dimensions.
After the resulting 2D pose has been aligned with the image plane's origin (this equals part of the normalization in \cite{drover18}), it is re-projected into three dimensions.
The resulting 3D pose $\widetilde{P}$ can then be compared to the original 3D pose $P$ in order to gain insights about the additional error made through normalization.
For this comparison, the same error model as in Protocol 2 is used.
This means the poses are scaled and translated before calculating the MPJPE.

In the following, let $P = [(X_1, Y_1, Z_1), \dotsc, (X_n, Y_n, Z_n)]$ be a 3D pose.
For simplicity we will assume that $dy = 0$, that is, the pose $P$ is only shifted along the x axis.

\subsubsection{Theoretical Analysis}
\label{sec:x-shift-error-theoretical}
Let $P_i = (X_i, Y_i, Z_i)$ be a joint of the pose $P$.
Then the coordinates of the projected point on the image plane are given by
\begin{equation}
	\label{eq:projected-point}
	x_i = f \frac{X_i}{Z_i} \ ,\enspace y_i = f \frac{Y_i}{Z_i} \ .
\end{equation}
Now, assume that $P_i$ is shifted by $dx$ along the x axis.
This results in a projected x coordinate of
\begin{equation}
	\label{eq:projected-x-y}
	x_i^\mathrm{S} = f \frac{X_i + dx}{Z_i} = f \frac{X_i}{Z_i} + f \frac{dx}{Z_i} = x_i + f \frac{dx}{Z_i}\ .
\end{equation}
The projected pose is then normalized by shifting all projected points such that the root joint is located in the origin of the image plane. 
Let the shifted root joint have coordinates $(dx, 0, Z)$.
Since $dy = 0$, each point is only shifted by $- f \frac{dx}{Z}$ along the x axis.
That means the x coordinate  $\widetilde{x}_i$ of the normalized point is given by
\begin{equation}
	\widetilde{x}_i
	= x_i^\mathrm{S} - f \frac{dx}{Z}
	= x_i + f \frac{dx}{Z_i} - f \frac{dx}{Z}
	= x_i + f dx (\frac{1}{Z_i} - \frac{1}{Z}) \ , 
\end{equation}
while the y coordinate is still the same as in \autoref{eq:projected-point}.
After $G$ estimates the depth $\widetilde{Z}_i$, the 2D point is re-projected into three dimensions using standard perspective projection. 
The resulting coordinates are 
\begin{align}
	\label{eq:re-projected-X}
	\widetilde{X}_i &= \frac{\widetilde{x}_i}{f} \cdot \widetilde{Z}_i
	= X_i \frac{\widetilde{Z}_i}{Z_i} + dx (\frac{1}{Z_i} - \frac{1}{Z}) \widetilde{Z}_i
	= \widetilde{Z_i} \cdot \left( \frac{X_i}{Z_i} + dx \left( \frac{1}{Z_i} - \frac{1}{Z} \right) \right) \ , \\
	\label{eq:re-projected-Y}
	\widetilde{Y}_i &= \frac{y_i}{f} \cdot \widetilde{Z}_i = \frac{Y_i}{Z_i} \cdot \widetilde{Z}_i \ .
\end{align}
The Euclidean distance (the Joint Position Error; JPE) of the re-projected point to the original point is given by
\begin{equation}
\label{eq:delta-d}
	\Delta d_i = \norm{ 
	\begin{pmatrix}
		\widetilde{X}_i - X_i \\
		\widetilde{Y}_i - Y_i \\
		\widetilde{Z}_i - Z_i
	\end{pmatrix}
	}_2
\end{equation}
With perfect estimation of $\widetilde{Z_i} = Z_i$, we have $\widetilde{Y}_i = Y_i$ and \autoref{eq:delta-d} together with \autoref{eq:re-projected-X} describes a linear relationship between the offset $\abs{dx}$ and the Joint Position Error $\Delta d_i$.
But since the generator's estimation of $\widetilde{Z_i}$ can vary for different offsets $dx$, the JPE can actually become smaller for other $\widetilde{Z_i} \neq Z_i$. 
In order to obtain a correct lower bound for the Joint Position Error, the $\widetilde{Z_i}$ for which $\Delta d_i$ is minimal has to be calculated.
So the function to be minimized is
\begin{align}
	\label{eq:minimum-distance}
	f(\widetilde{Z}_i) &= \left ( \widetilde{Z}_i \cdot \left( \frac{X_i}{Z_i} + dx \left( \frac{1}{Z_i} - \frac{1}{Z} \right) \right ) - X_i \right)^2 + \left ( \widetilde{Z_i} \cdot \frac{Y_i}{Z_i} - Y_i \right )^2 + \left ( \widetilde{Z}_i - Z_i \right ) ^2 \\
	&= \left ( \widetilde{Z}_i \cdot a - X_i \right)^2 + \left ( \widetilde{Z_i} \cdot b - Y_i \right )^2 + \left ( \widetilde{Z}_i - Z_i \right )^2 \ ,
\end{align}
where $a \coloneqq \left( \frac{X_i}{Z_i} + dx \left( \frac{1}{Z_i} - \frac{1}{Z} \right) \right )$ and $b \coloneqq \frac{Y_i}{Z_i}$ have been substituted for better readability.
In order to obtain the optimal value for $\widetilde{Z}_i$, $f$ is differentiated by $\widetilde{Z}_i$:
\begin{equation}
	\label{eq:derivative-minimum-distance}
	f'(\widetilde{Z}_i) = 2 \cdot \left ( \widetilde{Z}_i \left (a^2 + b^2 \right ) - a X_i - b Y_i + \widetilde{Z}_i - Z_i \right ) \ .
\end{equation}
Setting this equal to zero gives
\begin{align}
	0 &= f'(\widetilde{Z}_i^\ast) \\
	\Leftrightarrow \widetilde{Z}_i^\ast & = \frac{a X_i + b Y_i + Z_i}{1 + a^2 + b^2} \ .
	\label{eq:z_i-min}
\end{align}
With $f$ being a square function with respect to $\widetilde{Z}_i$, it is easy to see that $f$ has indeed a minimum at $\widetilde{Z}_i^\ast$ and that this is $f$'s global minimum.
With this $\widetilde{Z}_i^\ast$, the optimal $\widetilde{X}_i^\ast$ and $\widetilde{Y}_i^\ast$ from \autoref{eq:re-projected-X} and \autoref{eq:re-projected-Y} can now be calculated:
\begin{align}
	\label{eq:x_i-min}
	\widetilde{X}_i^\ast &= \widetilde{Z}_i^\ast \cdot a
	= \frac{a^2 X_i + a b Y_i +  a Z_i}{1 + a^2 + b^2} \ ,\\
	\label{eq:y_i-min}
	\widetilde{Y}_i^\ast &= \widetilde{Z}_i^\ast \cdot b
	= \frac{b^2 Y_i + a b X_i + b Z_i}{1 + a^2 + b^2} \ . 
\end{align}
Putting Equations \eqref{eq:z_i-min}, \eqref{eq:x_i-min} and \eqref{eq:y_i-min} into \autoref{eq:delta-d} yields a total minimal error of 
\Todo{Make roots look nice}
\begin{align}
	\Delta d_i &= 
	\sqrt{\left (\widetilde{X}_i^\ast - X_i \right )^2 + \left (\widetilde{Y}_i^\ast - Y_i \right )^2 + \left (\widetilde{Z}_i^\ast - Z_i \right )^2}\\
	& \stackrel{\footnotemark}{=} \sqrt{\frac{\left (a Y_i - b X_i \right )^2 + \left (a Z_i - X_i \right )^2 + \left (b Z_i - Y_i \right )^2}{1 + a^2 + b^2}} \\
	\label{eq:bzy-zero}
	&= \sqrt{\frac{\left (dx \cdot Y_i \cdot \left (\frac{1}{Z_i} - \frac{1}{Z} \right) \right)^2 + \left (a Z_i - X_i \right )^2}{1 + a^2 + b^2}}\\
	& = \sqrt{\frac{\left (dx \cdot Y_i \cdot \left (\frac{1}{Z_i} - \frac{1}{Z} \right ) \right)^2 + \left (dx \cdot Z_i \cdot \left (\frac{1}{Z_i} - \frac{1}{Z} \right ) \right)^2}{1 + a^2 + b^2}} \\
	& = \sqrt{\frac{dx^2 \cdot \left (\frac{1}{Z_i} - \frac{1}{Z} \right )^2 \cdot \left (Y_i^2 + Z_i^2 \right ) }{1 + a^2 + b^2}}\\
	\label{eq:minimum-delta-d}
	& = \abs{dx \left (\frac{1}{Z_i} - \frac{1}{Z} \right )} \sqrt{\frac{Y_i^2 + Z_i^2}{1 + a^2 + b^2}} \ .
\end{align}
In \autoref{eq:bzy-zero} $b Z_i - Y_i = 0$ has been used.
With this $\Delta d_i$ the MPJPE for the whole pose can then be calculated as the mean of all $\Delta d_i$s:
\footnotetext{This equation was derived using a computer algebra system.}
\begin{equation}
	\label{eq:minimum-mpjpe-on-shift}
	\Delta d = \frac{1}{n} \sum_{i = 1}^{n} \Delta d_i 
\end{equation}
In contrast to the linear relationship for $\widetilde{Z_i} = Z_i$, for optimal $\widetilde{Z_i}$, the minimal MPJPE converges to a constant for large $dx$.

This result shows that even with the generator estimating the depth of each point optimally, the estimated 3D pose will never match the original 3D pose up to shifting and scaling (the poses' root joints are already aligned and in general there exists no factor $\alpha$ such that the pose are equal).
Especially the re-projection of the 2D poses into three dimensional space without first shifting the 2D poses back as they were before normalization adds an essential part to the MPJPE.
In fact, the effects of 2D pose normalization by shifting only vanish if $dx = 0$ or $Z_i = Z$ for all $i$. 
In the latter case all joints of the pose are located in a plane parallel to the image plane.

The theoretical results above also show another downside of normalization by shifting.
In case the generator is trained with 2D poses which have multiple different offsets $dx$ and $dy$, ambiguities are introduced to the system:
As the generator cannot know whether the pose it receives was originally located in the origin of the image plane or not, it cannot correctly learn which depth offset belongs to which pose.
One easy solution for this second problem is to train only with poses that all have the same offsets in x and y direction (this has been done for the synthetic 2D poses in \autoref{sec:results-synthetic}).
As this is rarely the case in reality, that would make the system only viable for very specific applications.

For a lower bound with $dx \neq 0$ \emph{and} $dy \neq 0$, $b$ has to be substituted with $\left( \frac{Y_i}{Z_i} + dy \left( \frac{1}{Z_i} - \frac{1}{Z} \right) \right )$ in the above calculations.

\subsubsection{Experimental Results}

In order to verify the theoretical findings, experimental MPJPEs are calculated for poses with different offsets $dx$.
\autoref{fig:x-shift-error} (a) depicts a plot of the MPJPEs for synthetic 2D poses sampled from the 3D poses of Subjects 9 and 11 with different offsets in x direction (measured using Protocol 2), along with a plot of the minimal MPJPE calculated in the previous section in \autoref{eq:minimum-mpjpe-on-shift}.
As it turns out, both curves have the same approximate shape, but the sampled poses' MPJPE is significantly smaller than the theoretical MPJPE.
The difference between the two curves is increasing for larger offsets $\abs{dx}$.
At first glance, this seemingly contradicts the fact that the above $\Delta d$ represents a lower bound for the error.

However, one important detail has not been taken into account yet:
During the evaluation with Protocol 2, shifting and scaling are applied to the estimated poses. 
For the root joint $r$ we have $Z_r = Z$ in \autoref{eq:minimum-delta-d}, so $\Delta d_r = 0$. 
This means $(\widetilde{X}_r, \widetilde{Y}_r, \widetilde{Z}_r) = (X_r, Y_r, Z_r)$, that is, the root joints of both poses are already aligned.
It is not as easy to make a statement about the scale of the pose $\widetilde{P}$ from a theoretical point of view only though.
Experimental results show that the average limb length of $\widetilde{P}$ is increasing monotonously with increasing offset $\abs{dx}$.
That means for a fair comparison the pose $\widetilde{P}$ has to be re-scaled by a factor $\alpha_{\widetilde{P}}$ in all dimensions before the MPJPE is calculated:
\begin{equation}
	\Delta d_i = \norm{ 
	\begin{pmatrix}
		\alpha_{\widetilde{P}} \cdot \widetilde{X}_i - X_i \\
		\alpha_{\widetilde{P}} \cdot \widetilde{Y}_i - Y_i \\
		\alpha_{\widetilde{P}} \cdot \widetilde{Z}_i - Z_i
	\end{pmatrix}
	}_2 \ .
\end{equation}
For Protocol 2, $\alpha_{\widetilde{P}}$ is calculated such that the average limb length of $\alpha_{\widetilde{P}} \cdot \widetilde{P}$ is equal to the average limb length of the respective subject.

\begin{figure}
	\begin{subfigure}[]{.3\textwidth}
		\centering
		\begin{tikzpicture}[trim axis left]
		\begin{axis}[
			tiny,
			width=1.25\textwidth,
			xlabel={$dx$ [m]},
%			axis lines = middle,
    		y label style={at={(axis description cs:0.35,1.15)},rotate=-90,anchor=north},
			ylabel={MPJPE [mm]},
			grid=both,
			enlarge x limits = false,
			grid style={draw=gray!50},
			minor tick num = 1,
			tick style={draw=gray!50},
			xtick={-100,-50,0,50,100},
			extra x ticks={0},
			scaled x ticks = false,
			no markers,
			every axis plot/.append style={}
		]
		\addplot table [x=a, y=b, col sep=comma] {figures/plot_e_03_07_original_x_shift.csv};
		\addplot table [x=a, y=b, col sep=comma] {figures/plot_theoretical_x_shift.csv};
		\end{axis}
		\end{tikzpicture}
		\subcaption{unscaled}
	\end{subfigure}\hfill
	\begin{subfigure}[]{.3\textwidth}
		\centering
		\begin{tikzpicture}[trim axis left]
		\begin{axis}[
						tiny,
			width=1.25\textwidth,
			xlabel={$dx$ [m]},
			%			axis lines = middle,
			y label style={at={(axis description cs:0.35,1.15)},rotate=-90,anchor=north},
			ylabel={MPJPE [mm]},
			grid=both,
			enlarge x limits = false,
			grid style={draw=gray!50},
			minor tick num = 1,
			tick style={draw=gray!50},
			xtick={-100,-50,0,50,100},
			extra x ticks={0},
			scaled x ticks = false,
			no markers,
			every axis plot/.append style={}
		]
		\addplot table [x=a, y=b, col sep=comma] {figures/plot_e_03_07_original_x_shift.csv};
		\addplot table [x=a, y=b, col sep=comma] {figures/plot_theoretical_x_shift_scaled.csv};
		\end{axis}
		\end{tikzpicture}
		\subcaption{scaled}
	\end{subfigure}\hfill
	\begin{subfigure}[]{.3\textwidth}
		\centering
		\begin{tikzpicture}[trim axis left]
			\begin{axis}[
				tiny,
				width=1.25\textwidth,
				xlabel={$dx$ [m]},
				%			axis lines = middle,
				y label style={at={(axis description cs:0.35,1.15)},rotate=-90,anchor=north},
				ylabel={MPJPE [mm]},
				grid=both,
				enlarge x limits = false,
				grid style={draw=gray!50},
				minor tick num = 1,
				tick style={draw=gray!50},
				xtick={-3,-2,-1,0,1,2,3},
				extra x ticks={0},
				scaled x ticks = false,
				no markers,
				every axis plot/.append style={}
				]
				\addplot +[restrict expr to domain={\coordindex}{180:401}] table [x=a, y=b, col sep=comma] {figures/plot_e_03_07_original_x_shift.csv};
				\addplot +[restrict expr to domain={\coordindex}{180:401}] table [x=a, y=b, col sep=comma] {figures/plot_theoretical_x_shift_scaled.csv};
			\end{axis}
		\end{tikzpicture}
		\subcaption{scaled}
	\end{subfigure}
			

	
	\caption{Theoretical (red) and experimental (blue) MPJPEs for different values of $dx$.
		The original theoretical minimal MPJPE does not represent a lower bound (a), meanwhile the one with scaled joint coordinates does (b and c). 
		The theoretical errors are obtained by calculating $\Delta d$ for every pose in the Protocol 2 test data.
		The distance between the camera and the poses is approximately 5m.}
	\label{fig:x-shift-error}
	\unsure[inline]{Should the plots be larger? Are angles better than absolute offsets as the x axis?}
\end{figure}
 
A plot of the resulting scaled function for the minimal error is shown in Figures \ref{fig:x-shift-error} (b) and (c).
Here it becomes clear that \autoref{eq:delta-d} does indeed represent a lower bound for the MPJPE.

As shown in \autoref{sec:evaluation}, the MPJPE for the monocular 2D poses of the Human3.6M dataset is about 25mm higher than the one for the synthetically generated data.
With those poses' root joints mostly having non-zero x and y offsets, the phenomenon discussed above is likely to contribute to the error.
This motivates altering the network in order to tolerate shifted poses, which will be discussed in \autoref{sec:network-adjusting}.

It is also noticeable that the difference between the observed and the theoretical MPJPE is not constant.
For small deviations along the x axis the network seems to be able to compensate the error introduced by shifting.
As $dx$ increases, so does the difference between the observed and the theoretical results until it converges to a constant.

	
\subsection{Shifting along the Z Axis}
\label{sec:z-shift-error}

In this section, constraint (B) will be analyzed.
It states that a 2D pose's designated norm limb should have length $0.1$ when the pose is fed to the generator.
In case the constraint is not fulfilled from the start, there can be two reasons for this: 
an incorrect focal length $f$ of the camera, or an either too large or too small distance $Z$ between the camera and the 3D pose's root joint.
When $Z$ and $f$ are the same for all poses, no matter the exact values, all poses can be normalized with the same factor and the generator's learning behavior is not affected by the scaling, as it could just as well receive the non-normalized poses (ignoring numerical issues).

In case $Z$ is fixed for all poses and only the focal length of the camera varies, the poses can be normalized without any problems.
As the focal length is only a scale factor the projected points are multiplied with, dividing by it results in the same 2D pose as if $f$ has been correct in the first place.

On the other hand, when the focal length $f$ of the camera is fixed and the varying lengths of the norm limb are caused by a varying distance between the camera and the root joint, normalization by scaling will introduce an additional error.
The intuitional explanation behind this is that the relative angles between the projection rays become smaller with greater camera distance.
This makes the perspective projection more and more similar to a \emph{weak} perspective projection the more the camera distance increases.
As a simple example to demonstrate that this issue cannot be solved with scaling only, consider two points $A = (5, 5)$ and $B = (3, 10)$. 
With the same perspective projection depicted in \autoref{fig:camera-projection-setup}, we get the projected x coordinates $a = \frac{5}{5} = 1$ and $b = \frac{3}{10} = 0.3$.
Now consider the same points shifted by $5$ units in positive z direction, increasing the camera distance: $A' = (5, 10)$ and $B' = (3, 15)$.
The resulting x coordinates are $a' = \frac{5}{10} = 0.5$ and $b' = \frac{3}{15} = 0.2$.
There exists no factor $\gamma$ such that $a = \gamma a'$ \emph{and} $b = \gamma b'$.

This phenomenon also appears in photography, where it is known as lens compression (a misleading term, as it has nothing to do with the lens).
It manifests itself in the photographed objects moving closer together when the camera distance increases (this is also the case in the example above).

In the following section, the effect of this phenomenon on the MPJPE will be analyzed.

\subsubsection{Theoretical Analysis}
Again consider a point $P_i=(X_i, Y_i, Z_i) \in P$ and a camera with focal length $f$. Assume that $P_i$ is shifted by $dz$ along the z axis.
The coordinates of the projected point on the image plane are
\begin{align}
	x_i &= f \frac{X_i}{Z_i + dz} \ , \\
	y_i &= f \frac{Y_i}{Z_i + dz} \ .
\end{align}
The projected points are now scaled such that they have the same (arbitrary) scale, that is, a norm limb length of 0.1.
Let $\rho$ be the scaling factor to achieve this.
The scaled projected point is then given by
\begin{align}
		\widetilde{x}_i &= \rho \cdot x_i 
		= \rho \cdot f \frac{X_i}{Z_i + dz} \ , \\
		\widetilde{y}_i &= \rho \cdot y_i
		= \rho \cdot f \frac{Y_i}{Z_i + dz} \ .
\end{align}
After the generator estimates the depth $\widetilde{Z}_i$, the point is re-projected into three dimensional space:
\begin{align}
	\widetilde{X}_i &= \frac{\widetilde{x}_i}{f} \cdot \widetilde{Z}_i
	= \rho \cdot \frac{X_i}{Z_i + dz} \cdot \widetilde{Z}_i \ , \\
	\widetilde{Y}_i &= \frac{\widetilde{y}_i}{f} \cdot \widetilde{Z}_i
	= \rho \cdot \frac{Y_i}{Z_i + dz}\cdot \widetilde{Z}_i \ .
\end{align}
Like in \autoref{sec:x-shift-error-theoretical}, the Euclidean distance to the original point should be minimized with respect to $\widetilde{Z_i}$.
We consider the function $f$ following from \autoref{eq:delta-d}:
\begin{equation}
	f(\widetilde{Z}_i) = \left ( \widetilde{Z}_i \cdot \rho  \cdot \frac{X_i}{Z_i + dz}- X_i \right)^2 + \left ( \widetilde{Z}_i \cdot \rho \cdot \frac{Y_i}{Z_i + dz} - Y_i \right)^2
					 + \left ( \widetilde{Z}_i - Z_i \right)^2 \ .
\end{equation}
Let $a \coloneqq \rho \cdot \frac{X_i}{Z_i + dz}$ and $b \coloneqq \rho \cdot \frac{Y_i}{Z_i + dz}$, then one can obtain the same derivative as in \autoref{eq:derivative-minimum-distance}.
Again, this yields
\begin{equation}
	\widetilde{Z}_i^\ast = \frac{a X_i + b Y_i + Z_i}{1 + a^2 + b^2} \ ,
\end{equation}
and $\widetilde{X}_i^\ast$ and $\widetilde{Y}_i^\ast$ as in Equations \eqref{eq:x_i-min} and \eqref{eq:y_i-min}.
Then the JPE  is given by 
\begin{align}
	\label{eq:z-shift-error}
	\Delta d_i &= \sqrt{\frac{\left (a Y_i - b X_i \right )^2 + \left (a Z_i - X_i \right )^2 + \left (b Z_i - Y_i \right )^2}{1 + a^2 + b^2}} \\ 
	&= \sqrt{\frac{\left (\rho \cdot \frac{X_i}{Z_i + dz}\cdot Z_i - X_i \right )^2 + \left (\rho \cdot \frac{Y_i }{Z_i + dz}\cdot Z_i- Y_i \right )^2}{1 + a^2 + b^2}} \\
	&= \sqrt{\frac{\left (X_i^2 \left (\rho \cdot \frac{Z_i}{Z_i + dz} - 1 \right) \right )^2 + \left (Y_i^2 \left (\rho \cdot \frac{Z_i}{Z_i + dz}-1 \right ) \right )^2}{1 + a^2 + b^2}} \\
	&= \abs{1 - \rho \cdot \frac{Z_i}{Z_i + dz}}  \sqrt{\frac{X_i^2 + Y_i^2}{1 + a^2 + b^2}}\ .
\end{align}
The MPJPE $\Delta d$ of the whole pose $P$ can be calculated as the mean of all $\Delta d_i$s (\autoref{eq:delta-d}).

The reason for the existence of an additional MPJPE is the scale factor $\rho$.
As $\rho$ is calculated for the whole pose rather than for each joint separately, $\rho \cdot \frac{Z_i}{Z_i + dz}$ is usually \emph{not} equal to 1.
This is only the case if $\rho = \frac{Z_i + dz}{Z_i}$ for all $i$, which means $Z_i = Z_j$ for all $1 \leq i, j \leq n$.
In this case, all joints are aligned in the same x-y plane and the additional MPJPE vanishes.

Other than the results for shifting along an image plane axis, these results do not necessarily have to provide a lower bound for the MPJPE:
In theory, if the generator estimates the depths $\widetilde{Z}_i = Z_i + dz$ for all $i$, the estimated pose $\widetilde{P}$ equals $P$ up to a shift.
At the time of evaluation with Protocol 2, the root joints of the estimated and ground truth poses are aligned and the poses are identical.
That means that it is theoretically possible that the system still estimates perfect 3D poses from poses with $dz \neq 0$.
However, in the next section it will be shown that the observable MPJPE follows the here-obtained $\Delta d$ nonetheless.

In general, there exists no factor $\rho$ such that $x_i = \widetilde{x_i}$ for all $i$.
That means the shifted projected pose and the original projected pose usually do not match each other.
Thus, if the generator receives poses with different total depths, it cannot know with which camera distance these poses have been captured.
Again, this means ambiguities are introduced to the system.

\subsubsection{Experimental Results}
\label{sec:error-on-shift-experimental}

\begin{figure}
		\centering
		\begin{tikzpicture}[trim axis left]
		\begin{axis}[
		tiny,
		width=\textwidth,
		height=8cm,
		xlabel={$dx$ [m]},
		%			axis lines = middle,
		y label style={at={(axis description cs:0.12,1.08)},rotate=-90,anchor=north},
		ylabel={MPJPE [mm]},
		grid=both,
		enlarge x limits = false,
		grid style={draw=gray!50},
		minor tick num = 1,
		tick style={draw=gray!50},
		scaled x ticks = false,
		no markers,
		xtick={-3,-2,...,15},
		every axis plot/.append style={}
		]
		\addplot table [x=a, y=b, col sep=comma] {figures/plot_e_03_07_original_z_shift.csv};
		\addplot table [x=a, y=b, col sep=comma] {figures/plot_theoretical_z_shift.csv};
		\addplot[color=black!60!green] table [x=a, y=b, col sep=comma] {figures/plot_theoretical_z_shift_scaled.csv};
		\end{axis}
		\end{tikzpicture}
	\caption{Experimental (blue), theoretical (red) and scaled theoretical (green) MPJPEs for different values of $dz$.
		The theoretical errors are obtained by calculating $\Delta d$ for every pose in the Protocol 2 test data.
		The distance between the camera and the poses is approximately 5m.}
	\Todo[inline]{Add axes titles, use proper image.
	use pgfplots to plot data}
	\label{fig:z-shift-error}
\end{figure}

In order to compare the theoretical results with the actual MPJPEs, the latter are calculated for different offsets $dz$ (using Protocol 2).
Similar to the experimental evaluation in \autoref{sec:error-on-shift-experimental}, scaling with a per pose factor $\alpha_{\widetilde{P}}$ is applied to $\widetilde{X}_i^\ast$, $\widetilde{Y}_i^\ast$ and $\widetilde{Z}_i^\ast$.
The camera distance in the experiments for $dz = 0$ is approximately 5m.
The resulting plots are shown in \autoref{fig:z-shift-error}.

The experimental results confirm the theoretical findings.
Although the generator could estimate $\widetilde{Z_i}$s such that the MPJPE stays the same for all $dz$ ($\widetilde{Z_i} = Z_i$ for all $i$), the MPJPE increases for increasing $\abs{dz}$.
This meets the expectations, as otherwise the generator would have to be able to infer the absolute depth of the pose from a normalized 2D projection.
The theoretical results seem to provide a lower bound for the error, but because for all considered $dz$ the theoretical MPJPE does not exceed the minimal experimental MPJPE, this could just be coincidence.
The same shape of the two curves however suggests that the theoretical results do actually describe the behavior of the real world MPJPE.
Compared to the effect of shifting along an image plane axis, the MPJPE does not grow significantly for larger offsets $dz$.
For an offset of 20m, the MPJPE is about 5mm higher than the one without any offset.
For negative offsets (i.e. poses closer to the camera), a 5mm increase is already measurable at 2.4m.
It comes as no surprise that the effect is stronger for poses closer to the camera.
The smaller the camera distance, the more extreme the change in relative projection ray angles, up until the point where the pose moves behind the camera.

It is also notable that for $dz = 0$, the scaled theoretical MPJPE is already at 5mm.
The reason for this lies in the scaling applied with Protocol 2.
Because the poses are scaled to match the average limb length of the subject instead of the corresponding ground truth pose's, a minor error is introduced, as explained in \autoref{sec:protocol2}.

After all, these results show that poses having different absolute z coordinates than those the system was trained on cannot be estimated with equal fidelity.
However, due to the minor increase in MPJPE this effect might be negligible in most applications.